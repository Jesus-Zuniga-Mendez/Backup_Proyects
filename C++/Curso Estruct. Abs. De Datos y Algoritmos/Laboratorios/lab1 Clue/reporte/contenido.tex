\tableofcontents
\listoffigures
\newpage

\section{Introduccion.}
	En este laboratorio, el objetivo fue que los estudiantes aplicaran los conocimientos
de memoria dinámica, punteros y, en general, el conocimiento previo de los cursos anteriores
de programación, con la finalidad de desarrollar el juego de mesa Clue, también conocido como
Cluedo, pero en su versión para computadora. El laboratorio implica que los estudiantes apliquen
programación orientada a objetos, así como la implementación de las mecánicas básicas del juego
de mesa original.

Para desarrollar ambas versiones, se tomó como punto de partida la idea de utilizar
una base de datos que contiene toda la información necesaria, de este modo, el programa
podría ser más compacto y eficiente, pues, no tendría que desarrollarse una gran cantidad de
condicionales ni analizar múltiples patrones, sino, simplemente, dejar que el proceso de
automatización se encargue de analizar la información proporcionada.
\section{Funcionamiento del programa.}

	Este programa requiere la implementación de múltiples objetos, básicamente dos clases,
una llamada jugador que incluye los atributos: name, para almacenar el nombre del jugador una 
vez que se instancia; location, para proporcionar un identificador que indique dónde se encuentra
en el tablero del juego; active es una variable que se debe utilizar para indicar si se instació
la clase; deck, es un arreglo de string que permite almacenar las cartas que el método Sr Vladimir
proporciona de forma aleatoria; Además, contiene el método callme, utilizado para darle el 
nombre a los jugadores.

\subsection{Repartición y números pseudoaleatorios}

	Para la repartición de cartas se utiliza el método Sr. Vladimir, este debe recibir la
longitud del arreglo así como un arreglo del cual tomará cartas. Básicamente esta función toma
el arreglo del parámetro, y lo copia en otros dos arreglos para así verificar, de uno si las 
cartas ya fueron asignadas, y de otro las distribuye. Esto es logrado mediante dos ciclos que
iteran en las posiciones de los arreglos ingresados y luego en la posición de la baraja del 
jugador. Este método es fundamental, pues, sin él, no se podría dar inicio de forma automática
al juego.
	
	Otra función fundamental del juego es randgen, que se utiliza en múltiples partes del
código para generar un número pseudoaleatorio a partir de la fecha y hora del computador. Es 
utilizada para tirar dados, y cualquier procedimiento que requiera obtener algún número al azar.

\subsection{Tablero de juego}

	Uno de los requisitos básicos del juego es proporcionar la posibilidad del imprimir las
posiciones de los jugadores en el tablero, por lo tanto, se utilizó una serie de funciones y 
variables especiales para imprimir el tablero con colores, mejorando así la presentación del mismo
y la facilidad para diferenciar los diferentes lugares por los cuales los jugadores se movilizan.
Con ello, se creó la clase Tablero, que cuenta con el método que permite mover las fichas. Esta
clase cuenta con atributos tales como las dimensiones del tablero de juego, esto a partir de una
matriz, además, esta se debe crear de forma dinámica, reservando así, un espacio de memoria no 
fijo.
	El tablero cuenta con un método que permite imprimir el estado actual del juego, esto
gracias a dos ciclos que iteran entre columnas y filas, además de identificar si ya se llegó a 
la posición límite (El borde) del mismo, con el propósito de imprimir un salto de línea.
	En el respectivo archivo que contiene todo lo referente a tablero.h, se encuentra el 
constructor y el destructor del objeto. Además, contiene el método que permite asignar las
dimensiones del tablero, llamado "ObtenerDimensiones" que, abre el archivo que contiene el
de referencia, para luego recorrerlo y contar el tamaño; de esta forma, establece las 
dimensiones de forma dinámica y efectiva

 
\section{Conclusiones.}
Después de desarrollar la primera version de este programa se pueden obtener las siguientes conclusiones.
\begin{itemize}
	\item Fue posible implementar un programa en los lenguajes de programación, Python y C++,para determinar el nombre de los codones constituyentes de una cadena de ARN.
	\item Se logró, de manera satisfactoria, utilizar los conocimientos fundamentales de los cursos anteriores de programación, para desarrollar los algoritmos solicitados para la práctica de laboratorio.
	\item Los programas fueron desarrollados de la forma propuesta en grupo, mediante el uso de una base de datos, permitiendo, así, reducir la extensión del código.
\end{itemize}
