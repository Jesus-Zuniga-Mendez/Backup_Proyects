\tableofcontents
\listoffigures
\newpage
\section{Introducción}
Para este laboratorio el objetivo fue crear un programa tipo juego de trivia con el fin de aprender a utilizar punteros, a continuación se adjunta el código escrito en C con el que se  soluciono el problema, en el código se encuentra la documentación necesaria para comprender que lógica siguen las funciones presentes en el programa.
\newpage
\section{Resolución}

\begin{verbatim} 
//Universidad de Costa Rica
//Jesus Zuñiga Mendez
//Carne: B59084
//laboratorio 4: Juego de Trivia

//librerias
#include <stdlib.h>
#include <stdio.h>
#include <time.h>
#include <string.h>

//controlan los colores de impresion
#define ANSI_COLOR_RED     "\x1b[31m"
#define ANSI_COLOR_GREEN   "\x1b[32m"
#define ANSI_COLOR_YELLOW  "\x1b[33m"
#define ANSI_COLOR_BLUE    "\x1b[34m"
#define ANSI_COLOR_MAGENTA "\x1b[35m"
#define ANSI_COLOR_CYAN    "\x1b[36m"
#define ANSI_COLOR_RESET   "\x1b[0m"
//
//
//
//
//
//funcion que imprime la bienvenida
int Bienvenida(void){
    printf(ANSI_COLOR_CYAN 
"  ____   _                                 _      _          _  _  _ \n
        |  _ \\ (_)                               (_)    | |        | || || |\n
         | |_) | _   ___  _ __ __   __ ___  _ __   _   __| |  ___   | || || |\n
  |  _ < | | / _ \\| '_ \\\\ \\ / // _ \\| '_ \\ | | / _` | / _ \\  | || || |\n
        | |_) || ||  __/| | | |\\ V /|  __/| | | || || (_| || (_) | |_||_||_|\n
    |____/ |_| \\___||_| |_| \\_/  \\___||_| |_||_| \\__,_| \\___/  (_)(_)(_)\n" 
    ANSI_COLOR_RESET);                                                                   
    printf("\n");
    printf("\n");
    return 0;
}
//*************************************************************************************
//*************************************************************************************
//_____________________________________________________________________________________

//funcion que imprime el titulo de empate
int Empate(void){
    printf(ANSI_COLOR_CYAN 
"  ______                           _        \n
 |  ____|                         | |       \n
 | |__    _ __ ___   _ __    __ _ | |_  ___ \n
 |  __|  | '_ ` _ \\ | '_ \\  / _` || __|/ _ \\\n
 | |____ | | | | | || |_) || (_| || |_|  __/\n
 |______||_| |_| |_|| .__/  \\__,_| \\__|\\___|\n
                      | |                     \n
                      |_|   \n" 
                                             ANSI_COLOR_RESET);
    printf("\n");
    printf("\n");
    return 0;
}
//*************************************************************************************
//*************************************************************************************
//_____________________________________________________________________________________

//funcion que imprime el titulode ganador
int Ganador(void){
    printf(ANSI_COLOR_CYAN 
"   _____            _   _            _____    ____   _____  \n
 /  ____|    /\\    | \\ | |    /\\    |  __ \\  / __ \\ |  __ \\ \n
 | |  __    /  \\   |  \\| |   /  \\   | |  | || |  | || |__) |\n
 | | |_ |  / /\\ \\  | . ` |  / /\\ \\  | |  | || |  | ||  _  / \n
 | |__| | / ____ \\ | |\\  | / ____ \\ | |__| || |__| || | \\ \\ \n
\\ _____|/_/    \\_\\|_| \\_|/_/    \\_\\|_____/  \\____/ |_|  \\_\\" 
ANSI_COLOR_RESET);
    printf("\n");
    printf("\n");
    return 0;
}
//*************************************************************************************
//*************************************************************************************
//_____________________________________________________________________________________


//funcion que imprime el titulo
int Titulo(void){
    printf(ANSI_COLOR_CYAN 
"  _______  _____   _____ __      __ _____             _  _  _ \n
 |__   __||  __ \\ |_   _|\\ \\    / /|_   _|    /\\     | || || |\n
    | |   | |__) |  | |   \\ \\  / /   | |     /  \\    | || || |\n
    | |   |  _  /   | |    \\ \\/ /    | |    / /\\ \\   | || || |\n
    | |   | | \\ \\  _| |_    \\  /    _| |_  / ____ \\  |_||_||_|\n
    |_|   |_|  \\_\\|_____|    \\/    |_____|/_/    \\_\\ (_)(_)(_)\n"
                  ANSI_COLOR_RESET);  
    printf("\n");
    printf("\n");
    return 0;
}
//*************************************************************************************
//*************************************************************************************
//_____________________________________________________________________________________


//funcion que lee los nombres de los jugadores
// NumeroJugadores = LeerNombres(PArregloJugadores,4)
int LeerNombres(char* arreglo[])
{
    //recibimos un puntero como parametro para llenarlo con los nombres de los 
    jugadores
    //ciclo que controla que el usuario llene los datos de forma correcta
    //se ejecuta siempre que los usuarios sean mayores a 4 y sale de la funcion si
    //digita un uno
    int Jugadores = 5;
    while (Jugadores > 4)
    {
        printf(ANSI_COLOR_RESET "Digite el numero de jugadores (maximo 4)\n");
        scanf("%d",&Jugadores);
        //comprobamos que el usuario digite un numero correcto entre 2 y 4
        if (Jugadores > 4)
        {
            printf("\n");
            printf(ANSI_COLOR_RED "EL MAXIMO DE JUGADORES
             ES 4 (0 PARA SALIR)\n"ANSI_COLOR_RESET);
            printf("\n");
        }
        if (Jugadores == 1)
        {
            printf("\n");
            printf(ANSI_COLOR_RED "Este Juego empieza 
            a ser divertido cuando juegas con alguien\n
            Ve y busca un amigo (0 PARA SALIR)\n"ANSI_COLOR_RESET);
            Jugadores = 5;
            printf("\n");
        }

    }
    for(int i = 0; i < Jugadores; i++) {
        system("clear");
        Titulo();
        if (i==0){
            printf(ANSI_COLOR_BLUE "Digite el 
            Nombre de jugador 1\n" ANSI_COLOR_RESET);
            scanf("%s", arreglo[i]);
            printf("\n");
        }
        if (i==1){
            printf(ANSI_COLOR_YELLOW "Digite el 
            Nombre de jugador 2\n" ANSI_COLOR_RESET);
            scanf("%s", arreglo[i]);
            printf("\n");
        }
        if (i==2){
            printf(ANSI_COLOR_GREEN "Digite el 
            Nombre de jugador 3\n" ANSI_COLOR_RESET);
            scanf("%s", arreglo[i]);
            printf("\n");
        }
        if (i==3){
            printf(ANSI_COLOR_MAGENTA "Digite el 
            Nombre de jugador 4\n" ANSI_COLOR_RESET);
            scanf("%s", arreglo[i]);
            printf("\n");
        }
    }
    return Jugadores;
}

//*************************************************************************************
//*************************************************************************************
//_____________________________________________________________________________________


//funcion que aumenta en uno los puntos de un jugador
// SumarPunto(PPuntos[],0);
void SumarPunto(int arreglo[], int posicion)
{
    //recibimos un puntero como parametro para modificarlo
    int a = arreglo[posicion];
    a++;
    arreglo[posicion] = a;
}

//*************************************************************************************
//*************************************************************************************
//_____________________________________________________________________________________


//funcion que extrae una pregunta nueva
//PreguntaNueva(PArregloJugadores, NumeroPregunta);;
int PreguntaNueva(char** Pcontrol, char* arreglo[], int pregunta)
{
    //abrimos el archivo
    FILE* archivo;
    archivo = fopen(Pcontrol[1], "r");
    //declaracion de variables
    int Correcta = 0;
    char saltarLinea[128];
    char saltarLineaVacia[128];
    char caracter;
    //ciclo que se recorre hasta llegar a la pregunta pedida, el ciclo lo que hace es 
    saltar bloques de
    //lineas hasta llegar a la pedida
    for (int i=1; i<pregunta; i++){
        fscanf(archivo, "%[^\n]%*c", saltarLinea);
        fscanf(archivo, "%[^\n]%*c", saltarLinea);
        fscanf(archivo, "%[^\n]%*c", saltarLinea);
        fscanf(archivo, "%[^\n]%*c", saltarLinea);
        fscanf(archivo, "%[^\n]%*c", saltarLinea);
        fscanf(archivo, "%[^\n]%*c", saltarLinea);
        caracter = fgetc(archivo);
        
        //el siguiente cilo permite ignorar la linea 8, en caso de que tenga 
        algun tipo de dato innecesario
        //evita que el programa se caiga
        char salto = '\n';
        int contador = 1;
        while (contador > 0){
            if (caracter == salto){
                contador=0;
            }
            else{
                caracter = fgetc(archivo);
                contador=1;
            }
        }        
    }
    fscanf(archivo, "%[^\n]%*c", arreglo[0]);
    fscanf(archivo, "%[^\n]%*c", arreglo[1]);
    fscanf(archivo, "%[^\n]%*c", arreglo[2]);
    fscanf(archivo, "%[^\n]%*c", arreglo[3]);
    fscanf(archivo, "%[^\n]%*c", arreglo[4]);
    fscanf(archivo, "%d", &Correcta);
    fclose(archivo);
    return Correcta;
}

//*************************************************************************************
//*************************************************************************************
//_____________________________________________________________________________________

//funcion que devuelve un numero random que sera el jugador inicial
//codigo tomado de https://www.tutorialspoint.com/c_standard_library/
c_function_rand.htm
int JugadorRandom(int limite)
{
    int resultado; 
    time_t t;
    /* Intializes random number generator */
    srand((unsigned) time(&t));
    resultado = (rand() % limite);
    return resultado;
}

//*************************************************************************************
//*************************************************************************************
//_____________________________________________________________________________________

//funcion que controla el formato del archivo, devuelve el numero de lineas
int ComprobarArchivo(char** Pcontrol)
{
    //se abre el archivo
    FILE* archivo;
    archivo = fopen(Pcontrol[1], "r");
    //numeroLinea sera el numero de linea
    int numeroLinea = 0; 
    //caracter que se va a leer
    char caracter;
    //contador que controla el ciclo
    int contador = 0;
    //sera 1 si hay dos lineas vacias seguidas
    int error = 0;
    //controla el caracter que se ah leido, 0 si es salto de linea, 1 si es caracter
    int esCaracter = 0;
    //sirve para comprobar el orden de lo leido
    int caracterAnterior = 0;
    //variable para controlar el 
    char salto = '\n';
    //se ejecuta hasta llegar al final del archivo
    while (contador < 1) {
        caracterAnterior = esCaracter;
        caracter = fgetc(archivo);
        //verificamos que lo leido sea un salto de linea o un caracter
        //si es salto de linea evaluemos si esta en un lugar correcto, sino pasamos
        if (caracter == salto){
            //si lo leido anterirmente es un caracter no ha problema ya que no hay
             doble salto de linea
            if (caracterAnterior == 1){
                    esCaracter = 0;
                    numeroLinea++;
            }
            else{
                //leemos el siguiente caracter para verificar que no se trate de 
                otro salto de linea
                caracterAnterior = esCaracter;
                caracter = fgetc(archivo);
                //error si no hay al menos una pregunta
                if (numeroLinea < 6){
                    error = 100;
                }
                //error si hay doble salto de linea
                if (caracter == salto){
                    error = 200;
                    esCaracter=0;
                }else{
                    esCaracter=1;
                    numeroLinea++;
                }
                //error si el salto de linea se encuentra entre preguntas
                float modulo = (numeroLinea % 7);
                if (modulo != 0){
                    error = 300;
                }
            }
        }else{
            esCaracter=1;
        }
        if((caracterAnterior == 0) && (caracter == -1)){
            error = 500;
        }
        if(caracter == -1)
        {
            contador = 1;
            //printf("caracterAnterior: %i, caracter: %i\n",caracterAnterior,caracter);
        }
    }
    //se ajustan las lineas para que sean multiplo de 7
    numeroLinea++;
    numeroLinea++;
    int modulo = (numeroLinea % 7);
    int salir = 1;
    if (modulo !=0){
        error = 400;
    }
    if (error != 0){
        numeroLinea = 0;
        printf(ANSI_COLOR_RED "Error: %d\n"ANSI_COLOR_RESET,error);
    }

    fclose(archivo);
    return numeroLinea;
}

//*************************************************************************************
//*************************************************************************************
//_____________________________________________________________________________________



//funciones que declaran el ganador de la partida
void DeclararGanador(char* arregloJugadores[], int arregloPuntos[])
{
    printf(ANSI_COLOR_RESET);
    //int puntajes[4] = {10,11,10,10};
    int puntajes[4] = {arregloPuntos[0],arregloPuntos[1],
    arregloPuntos[2],arregloPuntos[3]};
    int empate = 0;
    int puntajeEmpate = 0;
    int mayor = puntajes[0];
    int puntajeA = 0;
    int puntajeB = 0;
    int imprimirGanador = 0;
    //sacamos el mayor puntaje de los cuatro y definimos si hay empates
    for (int i = 0; i < 4; i++){
        puntajeA = puntajes[i];
        for (int j = 0; j < 4; j++){
            puntajeB = puntajes[j];
            if ((puntajeA > puntajeB) && (puntajeA > mayor)){
                mayor = puntajeA;
            }
            if ((puntajeA == puntajeB) && (i != j) && (puntajeA != 0)){
                puntajeEmpate = puntajeA;
                empate++;
            }
        }
    }

    //se imprime el ganador
    if (mayor > puntajeEmpate){
        system("clear");
        Ganador();
        for (int i=0; i < 4;i++){
            puntajeA=puntajes[i];
            if (puntajeA == mayor){
                printf("El Ganador es: %s\nPuntos finales: %d\nFelicidades\n",
                arregloJugadores[i],mayor);
            }
        }
    }
    //se imprime los empates
    if (mayor == puntajeEmpate){
        system("clear");
        Empate();
        printf("Hay un Empate con %d puntos:\n",mayor);
        for (int i=0; i < 4;i++){
            puntajeA=puntajes[i];
            if (puntajeA == mayor){
                printf("Jugador %d : %s\n" ,(i+1) ,arregloJugadores[i]);
            }
        }
    }   
}

//*************************************************************************************
//*************************************************************************************
//_____________________________________________________________________________________



// argc: la cantidad de argumentos enviados por la CLI
// argv: los valores de los argument os enviados por la CLI
int main(int argc, char** argv) {
    //se imprime la bienvenida
    system("clear");
    Bienvenida();
    //comprobamos el archivo que venga en formato correcto y 
    obtenemos el numero de lineas
    int lineas = ComprobarArchivo(argv);
    int resultado = 0;
    //este if comprueba que el archivo de las preguntas este con formato 
    correcto sino imprimimos
     un error.
    if (lineas == 0){
        printf(ANSI_COLOR_RED  "ERROR EN EL ARCHIVO DE PREGUNTAS COMPRUEBE QUE 
        ESTÉ EN EL FORMATO CORRECTO\nLeer el archivo README 
        para mayor informacion\n" ANSI_COLOR_RESET);
    }else if (lineas <= 28){
        //se imprime la alerta de archivo corto
        printf(ANSI_COLOR_RED  "El Archivo solo contiene 4 preguntas, 
        se recomiendan que tenga más preguntas\n(0 para Salir y cargar otro 
        archivo o cualquier tecla para continuar))\n" ANSI_COLOR_RESET);
        char respuesta[128];
        scanf("%s", respuesta);
        //strcmp compara strings, en este caso se usa para saber que 
        digito el ususario
        resultado = strcmp(respuesta,"0");
    }else{
        resultado = 1;
    }
    if (resultado != 0){
        //se declaran los arreglos y variables que contendran los nombres y 
        puntos de los jugadores
        char jugadorUno[128] = {""};
        char jugadorDos[128] = {""};
        char jugadorTres[128] = {""};
        char jugadorCuatro[128] = {""};
        int puntosJugadorUno = 0;
        int puntosJugadorDos = 0;
        int puntosJugadorTres = 0;
        int puntosJugadorCuatro = 0;
        int numeroJugadores = 0;
        //punteros que apuntaran a cada uno de los nombres y puntos de los jugadores.
        int Ppuntos[4] = {puntosJugadorUno,puntosJugadorDos,puntosJugadorTres,
        puntosJugadorCuatro};
        char* PJugadores[4] = {jugadorUno,jugadorDos,jugadorTres,jugadorCuatro};
        //se declara el areglo y variables que contendra las preguntas y la respuesta
        char pregunta[128];
        char opcionA[128];
        char opcionB[128];
        char opcionC[128];
        char opcionD[128];
        int correcta = 0;
        //se declara el puntuero que apuntara a las pregunta 
        char* Ppregunta[5] = {pregunta,opcionA,opcionB,opcionC,opcionD};

        //parte del codigo donde todo se integra para realizar el juego
        //Llenamos la lista de jugadores
        numeroJugadores = LeerNombres(PJugadores);
        //calculamos el numero de preguntas que traera el archivo
        int cantidadPreguntas = (lineas / 7);
        int contadorPregunta = 1;
        //definimos el jugador inicial
        int jugador = JugadorRandom(numeroJugadores);
        //mientras existan preguntas se ejecuta el ciclo
        //se le resta uno para que empiece en 0
        numeroJugadores = numeroJugadores -1;
        while (contadorPregunta <= cantidadPreguntas){
            system("clear");
            Titulo();
            //formato de color para la impresion de las preguntas al jugador
            if (jugador == 0){
                printf(ANSI_COLOR_BLUE);
            }
            if (jugador == 1){
                printf(ANSI_COLOR_YELLOW);
            }
            if (jugador == 2){
                printf(ANSI_COLOR_GREEN);
            }
            if (jugador == 3){
                printf(ANSI_COLOR_MAGENTA);
            }
            //lineas tiene la cantidad de lineas del archivo
            //numeroJugadores tiene la cantidad de jugadores 
            //jugadorInicial tenfra el jugador que comenzara la partida
            correcta = PreguntaNueva(argv, Ppregunta, contadorPregunta);
            printf("Turno de: %s",PJugadores[jugador]);
            printf("\n");
            printf("Pregunta numero %d\n",contadorPregunta);
            printf("* %s\n",Ppregunta[0]);
            printf("1: %s\n",Ppregunta[1]);
            printf("2: %s\n",Ppregunta[2]);
            printf("3: %s\n",Ppregunta[3]);
            printf("4: %s\n",Ppregunta[4]);
            //printf("%d\n",correcta);
            int respuesta = 0;
            scanf("%d",&respuesta);
            if (respuesta == correcta){
                SumarPunto(Ppuntos,jugador);
            }
            //continuamos con el siguiente jugador
            jugador++;
            if (jugador > numeroJugadores){
                jugador = 0;
            }
            contadorPregunta++;
        }
        DeclararGanador(PJugadores,Ppuntos);    
    }
}





\end{verbatim}  


%\lstinputlisting{Trivia.c}



	%\lstinputlisting{Trivia.c}
	%\lstinputlisting[language=C]{Trivia.c}	
\newpage	
\section{Conclusión}
Se concluye que el uso de punteros en C es fundamental para el buen manejo de los datos, aunque al principio el uso de punteros puede resultar algo abstracto después de usarlo algún tiempo se puede aclarar mejor el uso de los mismos y se logra entender la importancia que estos tienen


